\documentclass[onehalfspacing]{article}

\usepackage{amsmath,amssymb,amsthm}
\usepackage{ wasysym }
\usepackage{ stmaryrd }
\usepackage{ mathpartir }
\usepackage{xcolor}
\usepackage{bussproofs}
\usepackage[left=3.00cm, right=3.00cm, top=3.00cm,bottom=3.0cm]{geometry}




\newtheorem{theorem}{Theorem}[section]
\newtheorem{corollary}[theorem]{Corollary}
\newtheorem{lemma}[theorem]{Lemma}
\newtheorem{proposition}[theorem]{Proposition}
\newtheorem{definition}[theorem]{Definition}
\newtheorem{example}[theorem]{Example}

\newcommand{\llb}{\llbracket}
\newcommand{\rrb}{\rrbracket}
\newcommand{\blue}[1]{{\color{blue}#1}}
\newcommand{\var}{\text{var}}


\author{Alexander Pluska}
\title{Proofs as Programs in Classical Logic\\Notes}

\begin{document}

\maketitle

\section{Plan}

Goal:
\begin{itemize}
	\item Extract program from resolution proofs of $\forall\exists$-sentences over inductive datatypes.
\end{itemize}

\noindent Steps:

\begin{itemize}
	\item Give extensions of G\"odel's System \textbf{T} and HAS + EM$_1$ + SK$_1$ to arbitrary inductive datatypes (possibly GADTs) and prove (or rather check) properties, i.e. strong normalization, cut-elimination, uniqueness of normal forms.
	\item Adapt the results of~\cite{aschieri2014interactive} to exhibit realizers in $\mathcal{F} + \Phi$ for cut-free proofs in the extended version of HAS + EM$_1$ + SK$_1$ of formulas $\forall x\exists y Pxy$ where $P$ is a predicate in the extended version of \textbf{T}.
	\item Adapt the iterative learning from~\cite{aschieri2014interactive} to extract $\lambda$-terms from realizers.
	\item Give a proof translation from resolution proofs to cut-free sequent calculus proofs (already done?).
\end{itemize}

\noindent Questions:
\begin{itemize}
	\item When does the translated proof require more than EM$_1$?
	\item Using the methods in~\cite{aschieri2014interactive} the extracted term will be in simple $\lambda$-calculus. Is is possible to obtain a term in $\mathcal{F}$? (Talk with Federico Aschieri)
	\item Can the predicates be defined in $\mathcal{F}$ instead of $\mathcal{T}$? (probably yes)
	\item What happens if we add non-inductive theories?
	
\end{itemize}

\pagebreak

\section{Extracting constructive content from resolution proofs}
\subsection{The superposition calculus}
First let us define the calculus from which we wish to extract programs. It will comprise a core set of rules used in the vampire theorem prover taken from~\cite{Kov_cs_2013}. Note that we pay special attention to the usually neglected (and highly non-constructive) part of CNF transformation and Skolemization. Also note that we neglect the simplification ordering which is necessary to formulate a strategy for proof search but not for our proof transformation.

\noindent\textbf{Resolution.}
\begin{center}
\AxiomC{$A\vee B$}
\AxiomC{$\neg A'\vee C$}
\BinaryInfC{$(B\vee C)\theta$}
\DisplayProof
\end{center}
where $\theta$ is an mgu of $A$ and $A'$.

\noindent\textbf{Factoring.}
\begin{center}
	\AxiomC{$A\vee A'\vee B$}
	\UnaryInfC{$(A\vee B)\theta$}
	\DisplayProof
\end{center}
where $\theta$ is an mgu of $A$ and $A'$.

\noindent\textbf{Superposition.}
\begin{center}
	\AxiomC{$l=r\vee B$}
	\AxiomC{$L[s]\vee C$}
	\BinaryInfC{$(L[r]\vee B\vee C)\theta$}
	\DisplayProof
	\AxiomC{$l=r\vee B$}
	\AxiomC{$t[s] = t'\vee C$}
	\BinaryInfC{$(t[r] = t'\vee B\vee C)\theta$}
	\DisplayProof
	\AxiomC{$l=r\vee B$}
	\AxiomC{$t[s]\neq t'\vee C$}
	\BinaryInfC{$(t[r]\neq t'\vee B\vee C)\theta$}
	\DisplayProof
\end{center}
where $\theta$ is an mgu of $l$ and $s$, $s$ is not a variable, $L[s]$ is not an equality literal.

\noindent\textbf{Equality Resolution.}
\begin{center}
	\AxiomC{$s\neq t\vee C$}
	\UnaryInfC{$C\theta$}
	\DisplayProof
\end{center}
where $\theta$ is an mgu of $s$ and $t$.

\noindent\textbf{Equality Factoring.}
\begin{center}
	\AxiomC{$s=t\vee s'=t'\vee C$}
	\UnaryInfC{$(s=t\vee t\neq t'\vee C)\theta$}
	\DisplayProof
\end{center}
where $\theta$ is an mgu of $s$ and $s'$.

First let us look at an example of the transformation we going to do. We shall look at a proof of $\forall x\exists y: f(y) = g(x)$ from $\forall x: f(x) = g(h(x))$ and $\forall x: h(h'(x)) = x$. The program we extract hopefully is $h'$.

\begin{align}
	[\text{axiom}]&&\forall x: f(x) = g(h(x))\\
	[\text{cnf 1}]&&f(x_0) = g(h(x_0))\\
	[\text{axiom}]&&\forall x: h(h'(x)) = x\\
	[\text{cnf 3}]&&h(h'(x_0)) = x_0\\
	[\text{negated conjecture}]&&\neg\forall x\exists y: f(y) = g(x)\\
	[\text{ennf 5}]&&\exists x\forall y: f(y)\neq g(x)\\
	[\text{choice axiom}]&&\exists x\forall y: f(y)\neq g(x)\rightarrow \forall x : f(x) \neq g(s_0)\\
	[\text{skolemization 6, 7}]&&\forall x: f(x)\neq g(s_0)\\
	[\text{cnf 8}]&&f(x_1)\neq g(s_0)\\
	[\text{superposition 9, 2}]&\{\}&g(s_0)\neq g(h(x_0))\\
	[\text{superposition 10, 4}]&\{x_0\mapsto h'(x_0)\}&g(x_0)\neq g(s_0)\\
	[\text{equality resolution 11}]&&\bot
\end{align}

We first use the well known trick of adding the original conjecture to the negated one from~\cite{Luckham_1971} to transform this into a classical proof of $\forall x\exists y: f(x) = g(y)$ rather than a refutation of its negation:
\setcounter{equation}{0}
\begin{align}
	[\text{axiom}]&&\forall x: f(x) = g(h(x))\\
	[\text{cnf 1}]&&f(x_0) = g(h(x_0))\\
	[\text{axiom}]&&\forall x: h(h'(x)) = x\\
	[\text{cnf 3}]&&h(h'(x_0)) = x_0\\
	[\text{\color{red}{tautology}}]&&\forall x\exists y: f(y) = g(x)\vee\neg\forall x\exists y: f(y) = g(x)\\
	[\text{ennf 5}]&&\forall x\exists y: f(y) = g(x)\vee\exists x\forall y: f(y)\neq g(x)\\
	[\text{choice axiom}]&&\exists x\forall y: f(y)\neq g(x)\rightarrow \forall x : f(x) \neq g(s_0)\\
	[\text{choice axiom}]&&\forall x(\exists y: f(y)= g(x)\rightarrow f(s_1(x)) = g(x))\\
	[\text{skolemization 6, 7}]&&\forall x\exists y: f(y) = g(x)\vee\forall x: f(x)\neq g(s_0)\\
	[\text{skolemization 9, 8}]&&\forall x: f(s_1(x)) = g(x)\vee\forall x: f(x)\neq g(s_0)\\
	[\text{cnf 8}]&&f(s_1(x_1)) = g(x_1)\vee f(x_0)\neq g(s_0)\\
	[\text{superposition 11, 2}]&\{\}&f(s_1(x_1)) = g(x_1)\vee g(s_0)\neq g(h(x_0))\\
	[\text{superposition 12, 4}]&\{x_0\mapsto h'(x_0)\}&f(s_1(x_1)) = g(x_1)\vee g(x_0)\neq g(s_0)\\
	[\text{equality resolution 13}]&\{x_0\to s_0\}&f(s_1(x_1)) = g(x_1)
\end{align}

Next we eliminate free variables by propagating substitutions:

\setcounter{equation}{0}
\begin{align}
	[\text{axiom}]&&\forall x: f(x) = g(h(x))\\
	[\text{cnf 1}]&&f(h'(x_0)) = g(h(h'(x_0)))\\
	[\text{axiom}]&&\forall x: h(h'(x)) = x\\
	[\text{cnf 3}]&&h(h'(x_0)) = x_0\\
	[\text{tautology}]&&\forall x\exists y: f(y) = g(x)\vee\neg\forall x\exists y: f(y) = g(x)\\
	[\text{ennf 5}]&&\forall x\exists y: f(y) = g(x)\vee\exists x\forall y: f(y)\neq g(x)\\
	[\text{choice axiom}]&&\exists x\forall y: f(y)\neq g(x)\rightarrow \forall x : f(x) \neq g(s_0)\\
	[\text{choice axiom}]&&\forall x(\exists y: f(y)= g(x)\rightarrow f(s_1(x)) = g(x))\\
	[\text{skolemization 6, 7}]&&\forall x\exists y: f(y) = g(x)\vee\forall x: f(x)\neq g(s_0)\\
	[\text{skolemization 9, 8}]&&\forall x: f(s_1(x)) = g(x)\vee\forall x: f(x)\neq g(s_0)\\
	[\text{cnf 8}]&&f(s_1(x_1)) = g(x_1)\vee f(h'(x_0))\neq g(s_0)\\
	[\text{superposition 11, 2}]&\{\}&f(s_1(x_1)) = g(x_1)\vee g(s_0)\neq g(h(h'(x_0)))\\
	[\text{superposition 12, 4}]&\{\}&f(s_1(x_1)) = g(x_1)\vee g(x_0)\neq g(s_0)\\
	[\text{equality resolution 13}]&\{x_0\to s_0\}&f(s_1(x_1)) = g(x_1)
\end{align}


\setcounter{equation}{0}
\begin{align}
	[\text{axiom}]&&\forall x: f(x) = g(h(x))\\
	[\text{cnf 1}]&&f(h'(s_0)) = g(h(h'(s_0)))\\
	[\text{axiom}]&&\forall x: h(h'(x)) = x\\
	[\text{cnf 3}]&&h(h'(s_0)) = s_0\\
	[\text{tautology}]&&\forall x\exists y: f(y) = g(x)\vee\neg\forall x\exists y: f(y) = g(x)\\
	[\text{ennf 5}]&&\forall x\exists y: f(y) = g(x)\vee\exists x\forall y: f(y)\neq g(x)\\
	[\text{choice axiom}]&&\exists x\forall y: f(y)\neq g(x)\rightarrow \forall x : f(x) \neq g(s_0)\\
	[\text{choice axiom}]&&\forall x(\exists y: f(y)= g(x)\rightarrow f(s_1(x)) = g(x))\\
	[\text{skolemization 6, 7}]&&\forall x\exists y: f(y) = g(x)\vee\forall x: f(x)\neq g(s_0)\\
	[\text{skolemization 9, 8}]&&\forall x: f(s_1(x)) = g(x)\vee\forall x: f(x)\neq g(s_0)\\
	[\text{cnf 8}]&&f(s_1(x_1)) = g(x_1)\vee f(h'(s_0))\neq g(s_0)\\
	[\text{superposition 11, 2}]&\{\}&f(s_1(x_1)) = g(x_1)\vee g(s_0)\neq g(h(h'(s_0)))\\
	[\text{superposition 12, 4}]&\{\}&f(s_1(x_1)) = g(x_1)\vee g(s_0)\neq g(s_0)\\
	[\text{equality resolution 13}]&\{\}&f(s_1(x_1)) = g(x_1)
\end{align}

Next we remove all the skolem constants in the conjecture-tautology by unification, propagate this change, and finally reinterpet superposition as composition to yield a valid intuitionistic proof:

\setcounter{equation}{0}
\begin{align}
	[\text{axiom}]&&\forall x: f(x) = g(h(x))\\
	[\text{instantiation 1}]&&f(h'(x_1)) = g(h(h'(x_1)))\\
	[\text{axiom}]&&\forall x: h(h'(x)) = x\\
	[\text{instantiation 3}]&&h(h'(x_1)) = x_1\\
	[\text{tautology}]&&f(h'(x_1)) = g(x_1)\Rightarrow f(h'(x_1)) = g(x_1)\\
	[\text{equality 5, 2}]&\{\}&g(x_1)= g(h(h'(x_1)))\Rightarrow f(h'(x_1)) = g(x_1)\\
	[\text{equality 6, 4}]&\{\}&g(x_1)= g(x_1)\Rightarrow f(h'(x_1)) = g(x_1)\\
	[\text{equality 13}]&\{\}&f(h'(x_1)) = g(x_1)
\end{align}

Let us look at a second example: Consider the sentences $\forall x : p(x_0, f(x_0))$, $\forall x_0, x_1: (p(x_0, x_1)\Rightarrow q(x_0, g(x_1))$, $\forall x_0, x_1: (q(x_0, x_1)\Rightarrow r(x_0, h(x_1)))$. We take a look at a proof of $\forall x_0\exists x_1: r(x_0, x_1)$ which hopefully gives us $h\circ g\circ f$. First consider the output of vampire

\setcounter{equation}{0}
\begin{align}
	[\text{axiom}] && \forall x_0: p(x_0,f(x_0))\\
	[\text{axiom}] && \forall x_0, x_1 : (p(x_0,x_1) \Rightarrow q(x_0,g(x_1)))\\
	[\text{axiom}] && \forall x_0, x_1 : (q(x_0,x_1) \Rightarrow r(x_0,h(x_1)))\\
	[\text{negated conjecture}] &&\neg\forall  x_0 : \exists x_1 : r(x_0,x_1)\\
	[\text{ennf transformation 2}] && \forall x_0, x_1 : (q(x_0,g(x_1)) \vee \neg p(x_0,x_1)) \\
	[\text{ennf transformation 3}] &&\forall x_0, x_1 : (r(x_0,h(x_1)) \vee \neg q(x_0, x_1)) \\
	[\text{ennf transformation 4}] && \exists x_0 \forall x_1 : \neg r(x_0, x_1)\\
	[\text{choice axiom}] && \exists x_0 \forall x_1 : \neg r(x_0, x_1) \Rightarrow \forall x_1 : \neg r(s_0,x_1)\\
	[\text{skolemisation 7,8}] && \forall x_1 : \neg r(s_0, x_1)\\
	[\text{cnf transformation 1}] && p(x_0,f(x_0))\\
	[\text{cnf transformation 5}] && q(x_1,g(x_2)) \vee \neg p(x_1, x_2)\\
	[\text{cnf transformation 6}] && r(x_3,h(x_4)) \vee \neg q(x_3, x_4)\\
	[\text{cnf transformation 9}] &&. \neg r(s_0,x_5)\\
	[\text{resolution 12, 13}] &\{x_3\mapsto s_0, x_5\mapsto h(x_4)\}& \neg q(s_0,x_4)\\
	[\text{resolution 14, 11}] &\{x_1\mapsto s_0, x_4\mapsto g(x_2)\}&\neg p(s_0,x_2)\\
	[\text{resolution 15, 10}] &\{x_0\mapsto s_0, x_2\mapsto f(s_0)\}& \bot
\end{align}

Again replace negated conjecture by tautology

\setcounter{equation}{0}
\begin{align}
	[\text{axiom}] && \forall x_0: p(x_0,f(x_0))\\
	[\text{axiom}] && \forall x_0, x_1 : (p(x_0,x_1) \Rightarrow q(x_0,g(x_1)))\\
	[\text{axiom}] && \forall x_0, x_1 : (q(x_0,x_1) \Rightarrow r(x_0,h(x_1)))\\
	[\text{tautology}] &&\forall  x_0 \exists x_1 : r(x_0,x_1)\vee \neg\forall  x_0 \exists x_1 : r(x_0,x_1)\\
	[\text{ennf transformation 2}] && \forall x_0, x_1 : (q(x_0,g(x_1)) \vee \neg p(x_0,x_1)) \\
	[\text{ennf transformation 3}] &&\forall x_0, x_1 : (r(x_0,h(x_1)) \vee \neg q(x_0, x_1)) \\
	[\text{ennf transformation 4}] && \forall  x_0 : \exists x_1 : r(x_0,x_1)\vee \exists x_0 \forall x_1 : \neg r(x_0, x_1)\\
	[\text{choice axiom}] && \exists x_0 : \forall x_1 : \neg r(x_0, x_1) \Rightarrow \forall x_1 : \neg r(s_0,x_1)\\
	[\text{choice axiom}] &&\forall x_0(\exists x_1: r(x_0, x_1)\Rightarrow r(x_0, s_1(x_0)))\\
	[\text{skolemisation 7,8}] && \forall  x_0 \exists x_1 : r(x_0,x_1)\vee \forall x_1 : \neg r(s_0, x_1)\\
	[\text{skolemisation 9,10}] && \forall  x_0 :  r(x_0, s_1(x_0))\vee \forall x_1 : \neg r(s_0, x_1)\\
	[\text{cnf transformation 1}] && p(x_0,f(x_0))\\
	[\text{cnf transformation 5}] && q(x_1,g(x_2)) \vee \neg p(x_1, x_2)\\
	[\text{cnf transformation 6}] && r(x_3,h(x_4)) \vee \neg q(x_3, x_4)\\
	[\text{cnf transformation 11}] && r(x_6, s_1(x_6))\vee\neg r(s_0, x_5)\\
	[\text{resolution 14, 15}] &\{x_3\mapsto s_0, x_5\mapsto h(x_4)\}&r(x_6, s_1(x_6))\vee \neg q(s_0,x_4)\\
	[\text{resolution 16, 13}] &\{x_1\mapsto s_0, x_4\mapsto g(x_2)\}&r(x_6, s_1(x_6))\vee\neg p(s_0,x_2)\\
	[\text{resolution 17, 12}] &\{x_0\mapsto s_0, x_2\mapsto f(s_0)\}&r(x_6, s_1(x_6))
\end{align}

We propagate the substitutions

\setcounter{equation}{0}
\begin{align}
	[\text{axiom}] && \forall x_0: p(x_0,f(x_0))\\
	[\text{axiom}] && \forall x_0, x_1 : (p(x_0,x_1) \Rightarrow q(x_0,g(x_1)))\\
	[\text{axiom}] && \forall x_0, x_1 : (q(x_0,x_1) \Rightarrow r(x_0,h(x_1)))\\
	[\text{tautology}] &&\forall  x_0 \exists x_1 : r(x_0,x_1)\vee \neg\forall  x_0 \exists x_1 : r(x_0,x_1)\\
	[\text{ennf transformation 2}] && \forall x_0, x_1 : (q(x_0,g(x_1)) \vee \neg p(x_0,x_1)) \\
	[\text{ennf transformation 3}] &&\forall x_0, x_1 : (r(x_0,h(x_1)) \vee \neg q(x_0, x_1)) \\
	[\text{ennf transformation 4}] && \forall  x_0 : \exists x_1 : r(x_0,x_1)\vee \exists x_0 \forall x_1 : \neg r(x_0, x_1)\\
	[\text{choice axiom}] && \exists x_0 : \forall x_1 : \neg r(x_0, x_1) \Rightarrow \forall x_1 : \neg r(s_0,x_1)\\
	[\text{choice axiom}] &&\forall x_0(\exists x_1: r(x_0, x_1)\Rightarrow r(x_0, s_1(x_0)))\\
	[\text{skolemisation 7,8}] && \forall  x_0 \exists x_1 : r(x_0,x_1)\vee \forall x_1 : \neg r(s_0, x_1)\\
	[\text{skolemisation 9,10}] && \forall  x_0 :  r(x_0, s_1(x_0))\vee \forall x_1 : \neg r(s_0, x_1)\\
	[\text{cnf transformation 1}] && p(x_0,f(x_0))\\
	[\text{cnf transformation 5}] && q(x_1,g(x_2)) \vee \neg p(x_1, x_2)\\
	[\text{cnf transformation 6}] && r(s_0,h(x_4)) \vee \neg q(s_0, x_4)\\
	[\text{cnf transformation 11}] && r(x_6, s_1(x_6))\vee\neg r(s_0,h(x_4))\\
	[\text{resolution 14, 15}] &\{\}&r(x_6, s_1(x_6))\vee\neg q(s_0,x_4)\\
	[\text{resolution 16, 13}] &\{x_1\mapsto s_0, x_4\mapsto g(x_2)\}&r(x_6, s_1(x_6))\vee\neg p(s_0,x_2)\\
	[\text{resolution 17, 12}] &\{x_0\mapsto s_0, x_2\mapsto f(s_0)\}&r(x_6, s_1(x_6))
\end{align}

\setcounter{equation}{0}
\begin{align}
	[\text{axiom}] && \forall x_0: p(x_0,f(x_0))\\
	[\text{axiom}] && \forall x_0, x_1 : (p(x_0,x_1) \Rightarrow q(x_0,g(x_1)))\\
	[\text{axiom}] && \forall x_0, x_1 : (q(x_0,x_1) \Rightarrow r(x_0,h(x_1)))\\
	[\text{tautology}] &&\forall  x_0 \exists x_1 : r(x_0,x_1)\vee \neg\forall  x_0 \exists x_1 : r(x_0,x_1)\\
	[\text{ennf transformation 2}] && \forall x_0, x_1 : (q(x_0,g(x_1)) \vee \neg p(x_0,x_1)) \\
	[\text{ennf transformation 3}] &&\forall x_0, x_1 : (r(x_0,h(x_1)) \vee \neg q(x_0, x_1)) \\
	[\text{ennf transformation 4}] && \forall  x_0 : \exists x_1 : r(x_0,x_1)\vee \exists x_0 \forall x_1 : \neg r(x_0, x_1)\\
	[\text{choice axiom}] && \exists x_0 : \forall x_1 : \neg r(x_0, x_1) \Rightarrow \forall x_1 : \neg r(s_0,x_1)\\
	[\text{choice axiom}] &&\forall x_0(\exists x_1: r(x_0, x_1)\Rightarrow r(x_0, s_1(x_0)))\\
	[\text{skolemisation 7,8}] && \forall  x_0 \exists x_1 : r(x_0,x_1)\vee \forall x_1 : \neg r(s_0, x_1)\\
	[\text{skolemisation 9,10}] && \forall  x_0 :  r(x_0, s_1(x_0))\vee \forall x_1 : \neg r(s_0, x_1)\\
	[\text{cnf transformation 1}] && p(x_0,f(x_0))\\
	[\text{cnf transformation 5}] && q(s_0,g(x_2)) \vee \neg p(s_0, x_2)\\
	[\text{cnf transformation 6}] && r(s_0,h(g(x_2))) \vee \neg q(s_0, g(x_2))\\
	[\text{cnf transformation 11}] && r(x_6, s_1(x_6))\vee\neg r(s_0,h(g(x_2)))\\
	[\text{resolution 14, 15}] &\{\}&r(x_6, s_1(x_6))\vee\neg q(s_0,g(x_2))\\
	[\text{resolution 16, 13}] &\{\}&r(x_6, s_1(x_6))\vee\neg p(s_0,x_2)\\
	[\text{resolution 17, 12}] &\{x_0\mapsto s_0, x_2\mapsto f(s_0)\}&r(x_6, s_1(x_6))
\end{align}

\setcounter{equation}{0}
\begin{align}
	[\text{axiom}] && \forall x_0: p(x_0,f(x_0))\\
	[\text{axiom}] && \forall x_0, x_1 : (p(x_0,x_1) \Rightarrow q(x_0,g(x_1)))\\
	[\text{axiom}] && \forall x_0, x_1 : (q(x_0,x_1) \Rightarrow r(x_0,h(x_1)))\\
	[\text{tautology}] &&\forall  x_0 \exists x_1 : r(x_0,x_1)\vee \neg\forall  x_0 \exists x_1 : r(x_0,x_1)\\
	[\text{ennf transformation 2}] && \forall x_0, x_1 : (q(x_0,g(x_1)) \vee \neg p(x_0,x_1)) \\
	[\text{ennf transformation 3}] &&\forall x_0, x_1 : (r(x_0,h(x_1)) \vee \neg q(x_0, x_1)) \\
	[\text{ennf transformation 4}] && \forall  x_0 : \exists x_1 : r(x_0,x_1)\vee \exists x_0 \forall x_1 : \neg r(x_0, x_1)\\
	[\text{choice axiom}] && \exists x_0 : \forall x_1 : \neg r(x_0, x_1) \Rightarrow \forall x_1 : \neg r(s_0,x_1)\\
	[\text{choice axiom}] &&\forall x_0(\exists x_1: r(x_0, x_1)\Rightarrow r(x_0, s_1(x_0)))\\
	[\text{skolemisation 7,8}] && \forall  x_0 \exists x_1 : r(x_0,x_1)\vee \forall x_1 : \neg r(s_0, x_1)\\
	[\text{skolemisation 9,10}] && \forall  x_0 :  r(x_0, s_1(x_0))\vee \forall x_1 : \neg r(s_0, x_1)\\
	[\text{cnf transformation 1}] && p(s_0, f(s_0))\\
	[\text{cnf transformation 5}] && q(s_0,g(f(s_0))) \vee \neg p(s_0, f(s_0))\\
	[\text{cnf transformation 6}] && r(s_0,h(g(f(s_0)))) \vee \neg q(s_0, g(f(s_0)))\\
	[\text{cnf transformation 11}] && r(x_6, s_1(x_6))\vee\neg r(s_0,h(g(f(s_0))))\\
	[\text{resolution 14, 15}] &\{\}&r(x_6, s_1(x_6))\vee\neg q(s_0,g(f(s_0)))\\
	[\text{resolution 16, 13}] &\{\}&r(x_6, s_1(x_6))\vee\neg p(s_0,f(s_0))\\
	[\text{resolution 17, 12}] &\{\}&r(x_6, s_1(x_6))
\end{align}

Then unification at 15 gives $s_1 = h\circ g\circ f$ and $s_0 = x_6$ and the final transformation gives: 


\setcounter{equation}{0}
\begin{align}
	[\text{axiom}] && \forall x_0: p(x_0,f(x_0))\\
	[\text{axiom}] && \forall x_0, x_1 : (p(x_0,x_1) \Rightarrow q(x_0,g(x_1)))\\
	[\text{axiom}] && \forall x_0, x_1 : (q(x_0,x_1) \Rightarrow r(x_0,h(x_1)))\\
	[\text{instantiation 1}] && p(x_6, f(x_6))\\
	[\text{instantiation 2}] && p(x_6, f(x_6))\Rightarrow q(x_6,g(f(x_6))) \\
	[\text{instantiation 3}] && q(x_6, g(f(x_6)))\Rightarrow r(x_6,h(g(f(x_6)))) \\
	[\text{tautology}] && r(x_6, h(g(f(x_6))))\Rightarrow r(x_6,h(g(f(x_6))))\\
	[\text{composition 6, 7}] &\{\}&q(x_6, g(f(x_6)))\Rightarrow r(x_6,h(g(f(x_6))))\\
	[\text{composition 5, 8}] &\{\}&p(x_6,f(x_6))\Rightarrow r(x_6, h(g(f(x_6))))\\
	[\text{composition 4, 9}] &\{\}&r(x_6, h(g(f(x_6))))
\end{align}

\pagebreak
\bibliographystyle{acm}
\bibliography{references}

\end{document}