\documentclass[a4paper,12pt]{article}

\title{Embedding intuitionistic into classical logic}
\author{Alexander Pluska}

\usepackage{amsmath,amssymb,amsthm}
\usepackage{ wasysym }
\usepackage{ stmaryrd }
\usepackage{ mathpartir }
\usepackage{bussproofs}
\usepackage{enumerate}
\usepackage{tikz-qtree}
\usepackage{stackengine}

\usepackage{hyperref}% http://ctan.org/pkg/hyperref
\usepackage{cleveref}% http://ctan.org/pkg/cleveref

\theoremstyle{definition}
\newtheorem{theorem}{Theorem}[section]
\theoremstyle{definition}
\newtheorem{corollary}[theorem]{Corollary}
\theoremstyle{definition}
\newtheorem{lemma}[theorem]{Lemma}
\theoremstyle{definition}
\newtheorem{proposition}[theorem]{Proposition}
\theoremstyle{definition}
\newtheorem{definition}[theorem]{Definition}
\theoremstyle{definition}
\newtheorem{example}[theorem]{Example}

\DeclareMathOperator*{\argmax}{arg\,max}

\begin{document}
	
	\maketitle
	
	\begin{abstract}
		The aim of this paper is for a propositional sequent to give a transformed sequent that is valid in classical logic if and only if the original one is valid in intuitionistic logic. Since intuitionistic propositional logic is decidable this is of course trivially possible, however our procedure will be local. This means that our transformation will also be sound for first-order logic, allowing us to reduce the non-intuitionistic content of such sequents which yields benefits for program extraction in a situation where only a classical prover is available. 
	\end{abstract}
	

	

	\section{Porpositional logic}
	
	Since the details of syntax are a delicate concern for proof theory we will first collect some important definitions and facts. Fix a countably infinite number of atomic propositions $A, B, C,\dots$ as well as a special atomic proposition $\bot$.
	
	\begin{definition}
		A \textit{Fomrula} can be constructed only in the following ways:
		\begin{itemize}
			\item every atomic proposition is a formula.
			\item given formulas $A, B$ we have formulas $A\wedge B, A\vee B, A\to B$.
		\end{itemize}
	 	With every formula $\varphi$ we can associate a unique labelled rooted binary tree, the \textit{syntax tree} $T(\varphi)$:
		 \begin{itemize}
		 	\item If $\varphi$ is atomic then $T(\varphi)$ consists of a single node with label $\varphi$: \tikz[scale=0.9, baseline=0mm, thick]{\node[draw] {$ \varphi$}}
		 	\item If $\varphi = \varphi_1\circ\varphi_2$ then the root node is again labelled with gamma and has two child nodes, at which the trees $T(\varphi_1)$ and $T(\varphi_2)$ are attached:
		 	\begin{center}
		 	\tikz[scale=1, baseline=-5mm, thick]{\Tree [.\node[draw]{$\varphi$}; [.$T(\varphi_1)$  ] [.$T(\varphi_2)$  ] ]}
	 		\end{center}
		 \end{itemize}
	 	$\psi$ is a \textit{subformula} of $\varphi$ if it is the label of some node of $T(\varphi)$.
	 \end{definition}
	 
	 
	 \begin{example}
	 	Consider the formula $((A\wedge B)\to A) \to (B\vee A)$. Then its syntax tree is
	 	\begin{center}
	 		\tikz[scale=1, baseline=-5mm, thick]{\Tree [.\node[draw]{$((A\to B)\to A)\to (B\vee A)$}; [.\node[draw]{$(A\to B)\to A$}; [.\node[draw]{$A\to B$}; [.\node[draw]{$A$}; ] [.\node[draw]{$B$}; ] ] [.\node[draw]{$A$}; ] ] [.\node[draw]{$B\vee A$}; [.\node[draw]{$B$}; ] [.\node[draw]{$A$};] ] ]}
	 	\end{center}
	 \end{example}
 
 	We are now ready to introduce our calculi for intuitionistic and classical logic:
	
	\begin{definition}
			A \textit{sequent} $A\Rightarrow B$ consists of two multisets of formulas $A, B$. The position of $A_i$ in $A_1,\dots, A_n\Rightarrow B_1,\dots, B_m$ is $0i$ and that of $B_i$ is $1i$. A formula $\varphi$ occurs at $p = (ji, p')$ if $\varphi$ occurs at $p'$ in $A_i$ if $j=0$ and in $B_i$ if $j=1$. The following rules make up our classical calculus $\mathbf{GC}$, corresponding roughly to the propositional fragment of \textbf{G3c} from~\cite{basicprooftheory}:\\
		\begin{center}
			\begin{tabular}{lll}
				\AxiomC{\hphantom{x}}
				\RightLabel{Ax}
				\UnaryInfC{$A,\Gamma\Rightarrow \Delta, A$}
				\DisplayProof&
				\AxiomC{\hphantom{x}}
				\RightLabel{L$\bot$}
				\UnaryInfC{$\bot,\Gamma\Rightarrow\Delta$}
				\DisplayProof&
				\\&&\\
				\AxiomC{$A, B,\Gamma\Rightarrow\Delta$}
				\RightLabel{L$\wedge$}
				\UnaryInfC{$A\wedge B, \Gamma\Rightarrow \Delta$}
				\DisplayProof&
				\AxiomC{$\Gamma\Rightarrow\Delta, A$}
				\AxiomC{$\Gamma\Rightarrow\Delta, B$}
				\RightLabel{R$\wedge$}
				\BinaryInfC{$\Gamma\Rightarrow \Delta, A\wedge B$}
				\DisplayProof&
				\\&&\\
				\AxiomC{$A, \Gamma\Rightarrow\Delta$}
				\AxiomC{$B, \Gamma\Rightarrow\Delta$}
				\RightLabel{L$\vee$}
				\BinaryInfC{$A\vee B, \Gamma\Rightarrow \Delta$}
				\DisplayProof&
				\AxiomC{$\Gamma\Rightarrow\Delta, A, B$}
				\RightLabel{R$\vee$}
				\UnaryInfC{$\Gamma\Rightarrow \Delta, A\vee B$}
				\DisplayProof&
				\\&&\\
				\AxiomC{$A\to B, \Gamma\Rightarrow\Delta, A$}
				\AxiomC{$B, \Gamma\Rightarrow\Delta$}
				\RightLabel{L$\to$}
				\BinaryInfC{$A\to B, \Gamma\Rightarrow \Delta$}
				\DisplayProof&
				\AxiomC{$A,\Gamma\Rightarrow\Delta, B$}
				\RightLabel{R$\to$}
				\UnaryInfC{$\Gamma\Rightarrow \Delta, A\to B$}
				\DisplayProof&
				\\&&\\
			\end{tabular}
		\end{center}
		
		To obtain our intuitionistic calculus \textbf{GI} (corresponding to \textbf{m-G3i} from~\cite{basicprooftheory}) substitute R$\to$ with
		\begin{center}
			\AxiomC{$A, \Gamma\Rightarrow B$}
			\RightLabel{R$\to'$}
			\UnaryInfC{$\Gamma\Rightarrow\Delta, A\to B$}
			\DisplayProof
		\end{center}
	
		A \textit{proof} of a formula $\varphi$ in a calculus $\mathbf{G}$ is a labelled binary rooted tree in which every node is labelled with a pair of the name of a rule in $\mathbf{G}$, its rule, and a sequent, its sequent, such that
		\begin{itemize}
			\item Every leaf is labelled with (Ax, $A,\Gamma\Rightarrow\Delta, A$) or (L$\bot$, $\bot, \Gamma\Rightarrow\Delta$) for some formula $A$, multi-sets of formulas $\Gamma,\Delta$.
			\item The sequent of an interior node can be derived from sequents of its child nodes according to its rule.
		\end{itemize}
		We usually present a proof directly as a sequence of applications of rules rather than in a more usual tree presentation. Note that we implicitly permute the antecedent and succedent, i.e. we implicitly may apply a number of exchange rules at any point in the proof. 
	\end{definition}

	\begin{example}
		The following is a proof of $(A\to B)\to A\Rightarrow B$:
		\begin{center}
			\AxiomC{}
			\RightLabel{Ax}
			\UnaryInfC{$A\to B, A\Rightarrow B, A$}
			\AxiomC{}
			\RightLabel{Ax}
			\UnaryInfC{$A, B\Rightarrow B$}
			\RightLabel{L$\to$}
			\BinaryInfC{$A\to B, A\Rightarrow B$}
			\DisplayProof
		\end{center}
	\end{example}

	\noindent We now recall some important properties of the given systems:
   
		
	\begin{theorem}[subformula property]
		Any formula that appears in a derivation is a subformula of a forumla that occurs in the root sequent.
	\end{theorem}
	
	We consider the following additional rules, which are not a priori part of our calculus:
		\begin{center}
		\begin{tabular}{lll}
			\AxiomC{$\Gamma\Rightarrow\Delta$}
			\RightLabel{Lweak}
			\UnaryInfC{$A,\Gamma\Rightarrow \Delta$}
			\DisplayProof&
			\AxiomC{$\Gamma\Rightarrow\Delta$}
			\RightLabel{Rweak}
			\UnaryInfC{$\Gamma\Rightarrow \Delta, A$}
			\DisplayProof&
			\\&&\\
			\AxiomC{$A, A,\Gamma\Rightarrow\Delta$}
			\RightLabel{Lcontr}
			\UnaryInfC{$A, \Gamma\Rightarrow \Delta$}
			\DisplayProof&
			\AxiomC{$\Gamma\Rightarrow\Delta, A, A$}
			\RightLabel{Rcontr}
			\UnaryInfC{$\Gamma\Rightarrow \Delta, A$}
			\DisplayProof&
			\\&&\\
		\end{tabular}
	
		\AxiomC{$\Gamma\Rightarrow A, \Delta$}
		\AxiomC{$\Gamma', A\Rightarrow \Delta'$}
		\RightLabel{Cut}
		\BinaryInfC{$\Gamma, \Gamma'\Rightarrow\Delta, \Delta'$}
		\DisplayProof
	\end{center}


	The following central result of proof theory will be useful to us in many places:
	\begin{theorem}[admissibility of cut (and structural rules)]
		The 5 above rules are admissible in our calculi for classical and intuitionistic logic, i.e. adding them we are not able to prove any additional theorems.
	\end{theorem}


	This means that we may assume that all proofs are free of these rules, but in constructing new proofs we may apply them. We exclude them from our calculus to keep it simple for proofs and because cut destroys the subformula property.

	\section{The Translation Procedure}
	
	\begin{definition}
		We say that two sequents $S, S'$ are \textit{equiprovable} if $S$ is intuitionistically provable iff $S'$ is intuitionistically provable and write $$S\sim_p S'.$$
	\end{definition}

	Clearly equiprovability is an equivalence relation. Since every sequent is either provable or not the above notion is trivial in a sense. However we can often argue that two sequents are equiprovable even when we cannot determine their individual provability. This is in particular useful when discussing transformations of sequents that are to preserve intuitionistic provability.
	
	\begin{example}
		The sequents $\Rightarrow A\vee B$ and $A\to C, B\to C\Rightarrow C$ are equiprovable. Note that they are not equivalent in the typical sense however, due to the absence of $C$ in the former. 
	\end{example}

		The reason that equivalence is insufficient for our procedure is that new atoms may be introduced. In trying to reverse double negation translation this is to be expected however: In double negation translation we collapse the veracity of $A$ and $\neg\neg A$, which are not intuitionistically equivalent. In our case we do the opposite, i.e. we must modifiy our formula such that $A$ and $\neg\neg A$ are no longer classically equivalent. For that we require new atoms.

		However there is clearly some connection between the two notions:
		
		\begin{lemma}\label{ep1}
			Let $A$ and $B$ be equivalent formulas, i.e. $A\Rightarrow B$ and $B\Rightarrow A$ are provable. Then \begin{enumerate}
				\item $A,\Gamma\Rightarrow\Delta \sim_p B, \Gamma\Rightarrow\Delta$
				\item $\Gamma\Rightarrow\Delta, A\sim_p\Gamma\Rightarrow\Delta, B$
			\end{enumerate}
		\end{lemma}
	
		\begin{proof}
			This is a direct consequence of the admissibility of cut, e.g.
			\begin{center}
				\AxiomC{$\vdots$}
				\noLine
				\UnaryInfC{$A,\Gamma\Rightarrow\Delta$}
				\AxiomC{$\vdots$}
				\noLine
				\UnaryInfC{$B\Rightarrow A$}
				\RightLabel{Cut}
				\BinaryInfC{$B, \Gamma\Rightarrow\Delta$}
				\DisplayProof
			\end{center}
		\end{proof}

		We observe some other basic properties of equiprovability.
	
	\begin{lemma}\label{ep2}
		Let $N$ be a atomic proposition that does not occur in $\Gamma$ or $\Delta$, $A, B$ arbitrary formulas that do not contain $N$, then:
		\begin{enumerate}
			\item $\Gamma\Rightarrow\Delta\sim_p\Gamma\Rightarrow\Delta, N$
			\item $\Gamma\Rightarrow\Delta, A\sim_p\Gamma, A\to N\Rightarrow \Delta, N$
			\item $\Gamma, A, B\Rightarrow\Delta\sim_p\Gamma, A\wedge B\Rightarrow\Delta$
			\item $\Gamma\Rightarrow A\vee B,\Delta\sim_p\Gamma\Rightarrow A, B,\Delta$
		\end{enumerate} 
	\end{lemma}

	\begin{proof}
		1. If $\Gamma\Rightarrow\Delta$ is provable then we obtain a proof of $\Gamma\Rightarrow\Delta, N$ through a simple application of weakening. Suppose on the other hand we have a proof of  $\Gamma\Rightarrow\Delta, N$. Then by the subformula property $N$ may only occur as a pure atom. In particular $N$ can never appear in the antecedent as it would occur in $\Gamma$ then. Since we may assume that our proof does not contain weakenings, $N$ has to be introduced as part of $\Delta$ in the axioms. Therefore removing $N$ from all succedents produces a valid proof of $\Gamma\Rightarrow\Delta$.
		
		2. If $\Gamma\Rightarrow\Delta, A$ is provable we obtain $\Gamma, A\to N\Rightarrow \Delta, N$ through a simple application of cut:
		\begin{center}
			\AxiomC{$\vdots$}
			\noLine
			\UnaryInfC{$\Gamma\Rightarrow\Delta, A$}
			\AxiomC{\hphantom{X}}
			\RightLabel{Ax}
			\UnaryInfC{$A, A\to N\Rightarrow P, A$}
			\AxiomC{\hphantom{X}}
			\RightLabel{Ax}
			\UnaryInfC{$N, A\Rightarrow N$}
			\RightLabel{L$\to$}
			\BinaryInfC{$A, A\to N\Rightarrow N$}
			\RightLabel{Cut}
			\BinaryInfC{$\Gamma, A\to N\Rightarrow\Delta, N$}
			\DisplayProof
		\end{center}
		Suppose on the other hand we have a proof of $\Gamma, A\to N\Rightarrow \Delta, N$. Consider the proof obtained by replacing every occurrence of $N$ with $A$. By induction on the proof height it can be checked that this constitutes a valid proof of $\Gamma, A\to A\Rightarrow\Delta, N$. We obtain $\Gamma\Rightarrow\Delta, A$ through an application fo Cut:
		\begin{center}
			\AxiomC{\hphantom{X}}
			\RightLabel{Ax}
			\UnaryInfC{$A \Rightarrow A$}
			\RightLabel{R$\to$}
			\UnaryInfC{$\Rightarrow A\to A$}
			\AxiomC{$\vdots$}
			\noLine
			\UnaryInfC{$\Gamma, A\to A\Rightarrow\Delta, A$}
			\RightLabel{Cut}
			\BinaryInfC{$\Gamma\Rightarrow\Delta, A$}
			\DisplayProof
		\end{center}
	
		3. Going from $\Gamma, A, B\Rightarrow\Delta$ to $\Gamma, A\wedge B\Rightarrow\Delta$ is a single $L\wedge$ step. The other direction is a simple application of Cut:
		
		\begin{center}
			\AxiomC{\hphantom{X}}
			\RightLabel{Ax}
			\UnaryInfC{$A, B \Rightarrow A$}
			\AxiomC{\hphantom{X}}
			\RightLabel{Ax}
			\UnaryInfC{$A, B \Rightarrow B$}
			\RightLabel{R$\wedge$}
			\BinaryInfC{$A, B \Rightarrow A\wedge B$}
			\AxiomC{$\vdots$}
			\noLine
			\UnaryInfC{$\Gamma, A\wedge B\Rightarrow\Delta$}
			\RightLabel{Cut}
			\BinaryInfC{$\Gamma, A, B\Rightarrow\Delta$}
			\DisplayProof
		\end{center}
		
		4. Analogous to 3.
	\end{proof}

	\begin{definition}
		A sequent $\Gamma\Rightarrow \Delta$ is \textit{implicational} if each formula $A$ in $\Gamma$ is of the form $C_A\to (D_A\to E_A)$, $(C_A\to D_A)\to E_A$, $C_A\to D_A$ or $C_A$ for some atomic $C_A, D_A, E_A$ and each formula $B$ in $\Delta$ is of the form $C_B$ or $C_B\to D_B$ for some atomic $C_B, D_B$.
	\end{definition}

	While at first this might seems quite restrictive we can give a translation procedure that transforms every sequent into an equiprovable implicational sequent - even one containing a single atom in the succedent - so in some sense this is a normal form for intuitionistic sequents. One nice property of implicational sequents is that their proofs consist of implicational sequents and contain only the rules Ax, L$\bot$, L$\to$ and R$\to$
	
	The translation steps are reminiscent of the definitions of connectors in second order logic. Consider the following rules, where $X$ is not atomic, $N$ is a new atom:
	\begin{center}
		\begin{tabular}{clcl}
			(1)&$\Gamma\Rightarrow \Gamma, A, B$&$\rightsquigarrow$&$\Gamma\Rightarrow\Gamma, A\vee B$\\
			(2)&$\Gamma\Rightarrow A\to B$&$\rightsquigarrow$&$\Gamma, A\Rightarrow B$\\
			(3)&$\Gamma\Rightarrow A\vee B$&$\rightsquigarrow$&$\Gamma, (A\vee B)\to N\Rightarrow N$\\
			(4)&$\Gamma\Rightarrow A\wedge B$&$\rightsquigarrow$&$\Gamma,(A\wedge B)\to N\Rightarrow N$\\
			
			(5)&$\Gamma, A\wedge B\Rightarrow C$&$\rightsquigarrow$&$\Gamma, A, B\Rightarrow C$\\
			(6)&$\Gamma, A\vee B\Rightarrow C$&$\rightsquigarrow$&$\Gamma, (A\to C)\to (B\to C)\to N\Rightarrow  N$\\
			(7)&$\Gamma, (A\wedge B)\to C\Rightarrow D$&$\rightsquigarrow$&$\Gamma, A\to (B\to C)\Rightarrow D$\\
			(8)&$\Gamma, (A\vee B)\to C\Rightarrow D$&$\rightsquigarrow$&$\Gamma, A\to C, B\to C\Rightarrow D$\\
			(9)&$\Gamma, A\to (B\wedge C)\Rightarrow D$&$\rightsquigarrow$&$\Gamma, A\to B, A\to C\Rightarrow D$\\
			(10)&$\Gamma, A\to (B\vee C)\Rightarrow D$&$\rightsquigarrow$&$\Gamma, A\to (B\to D)\to (C\to D)\to N\Rightarrow N$\\
			(11)&$\Gamma, X\to A\to B\Rightarrow C$&$\rightsquigarrow$&$\Gamma, X\to N, N\to (A\to B)\Rightarrow C$\\
			(12)&$\Gamma, A\to X\to B\Rightarrow C$&$\rightsquigarrow$&$\Gamma, X\to N, A\to (N\to B)\Rightarrow C$\\
			(13)&$\Gamma, A\to B\to X\Rightarrow C$&$\rightsquigarrow$&$\Gamma, N\to X, A\to (B\to N)\Rightarrow C$\\
			(14)&$\Gamma, (X\to A)\to B\Rightarrow C$&$\rightsquigarrow$&$\Gamma, N\to X, (N\to A)\to B\Rightarrow C$\\
			(15)&$\Gamma, (A\to X)\to B\Rightarrow C$&$\rightsquigarrow$&$\Gamma, X\to N, (A\to N)\to B\Rightarrow C$\\
			(16)&$\Gamma, (A\to B)\to X\Rightarrow C$&$\rightsquigarrow$&$\Gamma, N\to X, (A\to B)\to N\Rightarrow C$\\
		\end{tabular}
	\end{center}

	\begin{example}\hphantom{x}
	\begin{center}
		\begin{tabular}{ll}
			&$(A\to B)\to (C\vee D)\to E\Rightarrow F\vee G$\\
			$\rightsquigarrow$&$(A\to B)\to (C\vee D)\to E, F\to N, G\to N\Rightarrow N$\\
			$\rightsquigarrow$&$(A\to B)\to M, M\to (C\vee D)\to E, F\to N, G\to N\Rightarrow N$\\
			$\rightsquigarrow$&$(A\to B)\to M, M\to O\to E, (C\vee D)\to O, F\to N, G\to N\Rightarrow N$\\
			$\rightsquigarrow$&$(A\to B)\to M, M\to O\to E, C\to O, D\to O, F\to N, G\to N\Rightarrow N$\\
		\end{tabular}
	\end{center}
	\end{example}

	\begin{lemma}
		Iteratively applying the above transformations we can obtain from every sequent $S$ an equiprovable implicational sequent $S_\to$.
	\end{lemma}

	\begin{proof}[Proof]
		
		There are three things to check: 1. There is an applicable transformation for every non-implicational sequent. 2. The translation procedure terminates on every input. 3. The sequents involved in every transformation are equiprovable.
		
		1. Suppose $\Gamma\Rightarrow\Delta$ is not implicational. Then the succedent is not a single atom, or there is a formula in the antecedent that is not of the form $(C\to D)\to E, C\to (D\to E), C\to D$ or $D$ for some atoms $C, D, E$. If the succedent is not a single atom one of transformations (1)-(4) is applicable. If it is but the antecedent contains a formula which is not an implication (5) or (6) are. If that formula is an implication but one of its arguments is not then (7)-(10) are applicable. Finally if it is an implication of implications but the involved formulas are not atomic one of the rules (11)-(16) applies.

		2. We proceed by defining a well-founded order on the set of sequents $\mathcal S$ and showing that each of the transformations is reductive. Define the complexity $c(\varphi)$ of a formula as usual as $c(A) = 0$ for atoms $A$ and $c(A\circ B) = c(A) + c(B) + 1$ for $\circ\in\{\wedge,\vee,\to\}$. $c$ induces a partial order $<_c$ on formulas. We can extend this to a ordering $<_{DM}$ on finite multisets of formulas $\mathcal M$, also knows as the Dershowitz-Manna ordering as follows:
		
		$M <_{DM}N$ iff there exist $X, Y\in\mathcal{M}$ such that
		\begin{itemize}
			\item $X\subseteq N$ is non-empty,
			\item $M = (N \setminus X) \cup Y$,
			\item for all $y\in Y$ there exists $x\in X$ such that $y <_S x$.
		\end{itemize}
		Since $<_c$ is well-founded so is $<_{DM}$~\cite{Dershowitz-Manna_1979}. Now consider the lexicographic order $\ll$ on $\mathcal{S}^5$ induced by the following orders:
		\begin{enumerate}[(i)]
			\item $(\Gamma\Rightarrow\Delta)\ll(\Gamma'\Rightarrow\Delta')$ iff $|\Delta| < |\Delta'|$ 
			\item $(\Gamma\Rightarrow\Delta)\ll(\Gamma'\Rightarrow\Delta')$ iff $\Delta <_{DM} \Delta'$
			\item $(\Gamma\Rightarrow\Delta)\ll(\Gamma'\Rightarrow\Delta')$ iff the number of occurrences of $\wedge$ in $\Gamma$ is lower than in $\Gamma' $.
			\item $(\Gamma\Rightarrow\Delta)\ll(\Gamma'\Rightarrow\Delta')$ iff the number of occurrences of $\vee$ in $\Gamma$ is lower than in $\Gamma' $.
			\item $(\Gamma\Rightarrow\Delta)\ll(\Gamma'\Rightarrow\Delta')$ iff $\Gamma <_{DM}\Gamma $.
		\end{enumerate}
		Then the diagonal of $\ll$ is our desired order $<$. Since each involved order is well-founded so is $\ll$ and therefore also $<$. Now every transformation is reductive with regard to $<$, i.e. (1) is reductive w.r.t (i), (2)-(4) are reductive w.r.t. (ii), (5), (7), (9) are reductive w.r.t (iii), (6), (8), (10) w.r.t 4 (iv) and (11) - (16) w.r.t. (v) and non are increasing with any of the higher orders. This proves termination.
		 
		3. For most cases equiprovability follows from Lemmas \ref{ep1} and \ref{ep2}, e.g. consider $$\Gamma, A\to (B\wedge C)\Rightarrow D\indent\rightsquigarrow\indent\Gamma, A\to B, A\to C\Rightarrow D$$Then we have
		\begin{align*}
			\Gamma, A\to (B\wedge C)\Rightarrow D&\overset{\ref{ep1}.1}{\sim_p}\Gamma, (A\to B)\wedge (A\to C)\Rightarrow D\\&\overset{\ref{ep2}.3}{\sim_p}\Gamma, A\to B, A\to C\Rightarrow D
		\end{align*}
		
		Some cases require simple applications of cut using intuitionistic tautologies, e.g. for
		$$\Gamma, A\vee B\Rightarrow C\indent\rightsquigarrow\indent\Gamma, (A\to C)\to (B\to C)\to N\Rightarrow  N$$
		we have 
		\begin{align*}
			\Gamma, A\vee B\Rightarrow C&\overset{\star}{\sim_p}\Gamma\Rightarrow (A\vee B)\to C\\&\overset{\ref{ep2}.3}{\sim_p}\Gamma\Rightarrow (A\to C)\wedge(B\to C)\\&\overset{\ref{ep2}.2}{\sim_p} \Gamma, ((A\to C)\wedge (B\to C))\to N\Rightarrow N\\&\overset{\ref{ep1}.1}{\sim_p}\Gamma, (A\to C)\to(B\to C)\to N\Rightarrow N.
		\end{align*}
		Where one direction of $\star$ is trivial and the other is obtained via cut with $$A\vee B, (A\vee B)\to C\Rightarrow C$$
		
		For the last 6 rules one direction in obtained via cut and the other one by substituting all occurrences of $N$ in a proof with $X$.
	\end{proof}
	
	This reduces our task to embedding just the implicational sequents from intuitionistic into classical logic. And we shall immediately present this embedding, even though showing its correctness will bee a bit involved:
	
	 Consider some implicational sequent $S := \Gamma\Rightarrow \Delta$. Let $\{(C_i\to D_i)\to E_i\}_{i\in I}$ be an enumeration of sequents of that form occurring in $\Gamma$ and $\{C_j\to D_j\}_{j\in J}$ of such sequents in $\Delta$, and let $i = j$ iff. For each atom $A\neq\bot$ occurring in $S$ and $i\in I\cup J$ define a new atom $A^{(i)}$ with the stipulation that $A^{(i)} = A^{(j)}$ if $C^{(i)} = C^{(j)}$ and $D^{(i)} = D^{(j)}$. Furthermore have $\bot^{(i)} = \bot$. We call all such $A^{(i)}$ primed. Then define 
	\begin{align*}
		\Gamma' = &\: (\Gamma\setminus\{(C_{i}\to D_i)\to E_i\:|\:i\in I\})
		\\&\cup \{(C_{i}^{(i)}\to D_i^{(i)})\to E_i\:|\:i\in I\}
		\\&\cup \{A_i^{(i)}\to B_i^{(i)}\:|\: A, B\text{ atomic}, A\to B\in\Gamma, i\in I\cup J\}
		\\&\cup \{C^{(i)}\to (D^{(i)}\to E^{(i)})\:|\: C, D, E\text{ atomic}, C\to (D\to E)\in\Gamma, i\in I\cup J\}
		\\&\cup \{D_i\to D_i^{(i)}\:|\:i\in I\cup J\}
	\end{align*}
as well as
\begin{align*}
	\Delta' = &\:(\Delta\setminus\{C_{j}\to D_j\:|\:j\in J\})
	\\&\cup \{C_j^{(j)}\to D_j^{(j)}\:|\: C, D\text{ atomic}, C\to D\in\Delta, j\in J\}
\end{align*}
and have $$S' := \Gamma' \Rightarrow \Delta' $$

	
	We prove that $S$ is intuitionistically valid iff $S'$ is classically valid by first showing that $S$ and $S'$ are equiprovable and then that $S'$ is provable intuitionistically iff it is classically provable.
	
	\begin{lemma}
		Any implicational sequent $S$ is equiprovable with $S'$.
	\end{lemma}
	
	\begin{proof}
		
		Suppose $S'$ is provable. Consider the proof obtained by replacing every occurrence of primed atom with its regular version, then a number of contractions and cuts with sequents of the form $\Rightarrow A\to A$ yields a proof of $S$. The proof of the other direction is much more involved.
		
		Suppose $S$ is provable. We show that $S'$ is provable by induction on the height of the proof of $S$. If the proof of $S$ consists of a single Ax/L$\bot$ then $S'$ can also be proven by a single Ax/L$\bot$. Suppose that last rule in the proof of $S$ is $L\to$, i.e. it ends in
		
		\begin{center}
			\AxiomC{$A\to B, \Gamma\Rightarrow\Delta, A$}
			\AxiomC{$B, \Gamma\Rightarrow\Delta$}
			\RightLabel{L$\to$}
			\BinaryInfC{$A\to B, \Gamma\Rightarrow \Delta$}
			\DisplayProof
		\end{center}
		
		There are three cases:
		
		1. $A, B$ are atoms. By induction hypothesis $A\to B, (A^{(i)}\to B^{(i)})_{i\in I}, \Gamma' \Rightarrow \Delta', A$ and $B,\Gamma' \Rightarrow \Delta'$ are provable and thus by weakening also $B, (A^{(i)}\to B^{(i)})_{i\in I}, \Gamma' \Rightarrow \Delta'$ and we have
		\begin{center}
			\AxiomC{$A\to B, \{A^{(i)}\to B^{(i)}\}_{i\in I},\Gamma' \Rightarrow\Delta' , A$}
			\AxiomC{$B, \{A^{(i)}\to B^{(i)}\}_{i\in I}, \Gamma' \Rightarrow\Delta' $}
			\RightLabel{L$\to$}
			\BinaryInfC{$A\to B, \{A^{(i)}\to B^{(i)}\}_{i\in I}, \Gamma' \Rightarrow \Delta' $}
			\DisplayProof
		\end{center}
		so $S'$ is indeed provable.
		
		2. $A$ atomic, $B = C\to D$ with $C, D$ atomic. Again by induction hypothesis we obtain that $A\to (C\to D), \{A^{(i)}\to(C^{(i)}\to D^{(i)})\}_{i\in I}, \Gamma'\Rightarrow \Delta', A$ and $C\to D, \{C^{(i)}\to D^{(i)}\}_{i\in I}, \Gamma'\Rightarrow\Delta'$ are provable. Since $\Gamma'$ contains $A\to A^{(i)}$ a cut followed by a contraction gives that $A\to (C\to D), \{A^{(i)}\to(C^{(i)}\to D^{(i)})\}_{i\in I}, \Gamma'\Rightarrow\Delta', A^{(i)}$ is provable for each $i$ and a cut with $C\to D\Rightarrow A\to (C\to D)$ gives that $C\to D, \{A^{(i)}\to(C^{(i)}\to D^{(i)})\}_{i\in I}, \Gamma'\Rightarrow\Delta', A^{(i)}$ is provable. Then a number of cuts with $C^{(i)}\to D^{(i)}\Rightarrow A^{(i)}\to (C^{(i)}\to D^{(i)})$ gives for each $j\in I=\{1,\dots,n\}$ (w.l.o.g. since $I$ was just chosen as just some arbitrary finite index set) $$C\to D, \{A^{(i)}\to(C^{(i)}\to D^{(i)})\}_{i\leq j}, \{C^{(i)}\to D^{(i)}\}_{i> j}, \Gamma'\Rightarrow\Delta', A^{(i)}$$ Note that for each $j$ we have
		
		\begin{center}
			\hspace*{-.7cm}
			\AxiomC{$\stackanchor{$\{A^{(i)}\to(C^{(i)}\to D^{(i)})\}_{i\leq j},$}{$C\to D,\{C^{(i)}\to D^{(i)}\}_{i> j},$} \Gamma'\Rightarrow\Delta', A^{(j)}$}
			\AxiomC{$\stackanchor{$\{A^{(i)}\to(C^{(i)}\to D^{(i)})\}_{i< j},$}{$C\to D,\{C^{(i)}\to D^{(i)}\}_{i\geq j}$}\Gamma'\Rightarrow\Delta'$}
			\RightLabel{L$\to$}
			\BinaryInfC{$\{A^{(i)}\to(C^{(i)}\to D^{(i)})\}_{i< j+1},C\to D,\{C^{(i)}\to D^{(i)}\}_{i\geq j+1},\Gamma'\Rightarrow\Delta'$}
			\DisplayProof
		\end{center}
		and we have already established the right argument for $j=0$ and the left argument for every $j\in I$. Therefore applying the above iteratively $n$ times we obtain $C\to D, \{A^{(i)}\to(C^{(i)}\to D^{(i)})\}_{i\in I}\Gamma'\Rightarrow\Delta'$ and finally 
		\begin{center}
			\AxiomC{$\stackanchor{$A\to (C\to D),\Gamma',$}{$\{A^{(i)}\to(C^{(i)}\to D^{(i)})\}_{i\in I},$} \Rightarrow \Delta', A$}
			\AxiomC{$\stackanchor{$C\to D, \Gamma',$}{$\{A^{(i)}\to(C^{(i)}\to D^{(i)})\}_{i\in I}$}\Rightarrow \Delta' $}
			\RightLabel{L$\to$}
			\BinaryInfC{$A\to (C\to D), \{A^{(i)}\to(C^{(i)}\to D^{(i)})\}_{i\in I}, \Gamma' \Rightarrow \Delta' $}
			\DisplayProof
		\end{center}
		which shows that $S'$ is indeed provable.
		
		3. $A = C\to D$  with $C, D$ atomic, $B$ atomic. Then by induction hypothesis $(C^{(i)}\to D^{(i)})\to B, \Gamma'\Rightarrow\Delta', C^{(j)}\to D^{(j)}$ and $B, \Gamma'\Rightarrow\Delta'$ are provable, where $C^{(i)} = C^{(j)}$ and $D^{(i)} = D^{(j)}$ so a single application of L$\to$ gives us $S'$.
		
		Finally suppose the last rule of the proof is R$\to$, i.e.
		
		\begin{center}
			\AxiomC{$A,\Gamma\Rightarrow B$}
			\RightLabel{R$\to$}
			\UnaryInfC{$\Gamma\Rightarrow \Delta, A\to B$}
			\DisplayProof
		\end{center}
		Then $A, B$ are atoms. Let $i$ be some new index. Let $\Gamma''$ be $\Gamma$ where every non-primed atom $X$ has been renamed to $X^{(i)}$. Then $A^{(i)}, \Gamma''\Rightarrow B^{(i)}$ is provable and thus so is $\bigwedge\Gamma''\Rightarrow A^{(i)}\to B^{(i)}$. Since $\Gamma'\Rightarrow \bigwedge\Gamma''$ is a tautology a single cut yields that $S'$ is provable.
	\end{proof}


	\begin{lemma}
		Let $S$ be an implicational sequent such that $S'$ is provable classically. Then $S'$ is provable intuitionistically.
	\end{lemma}

	\begin{proof}
		We proceed by induction on the proof height of the classical proof of $S'$. If the last rule is Ax or L$\bot$ (or L$\to$) the proof is already intuitionistically valid (or it follows directly from the induction hypothesis). Otherwise the proof ends in $n$ applications of R$\to$ where $n$ is chosen maximally, i.e.
		\begin{center}
			\AxiomC{}
			\RightLabel{K}
			\UnaryInfC{$A_1^{(1)},\dots, A_n^{(n)},\Gamma'\Rightarrow\Delta', B_1^{(1)},\dots B_n^{(n)}$}
			\RightLabel{R$\to$}
			\UnaryInfC{\vdots}
			\RightLabel{R$\to$}
			\UnaryInfC{$\Gamma'\Rightarrow\Delta', A_1^{(1)}\to B_1^{(1)},\dots, A_n^{(n)}\to B_n^{(n)}$}
			\DisplayProof
		\end{center}
		with $\text{K }\neq\text{ R}\to$. If K $=$ L$\bot$ we have $\bot\in\Delta'$ and therefore we immediately have
		\begin{center}
			\AxiomC{}
			\RightLabel{L$\bot$}
			\UnaryInfC{$\Gamma'\Rightarrow\Delta', A_1^{(1)}\to B_1^{(1)},\dots, A_n^{(n)}\to B_n^{(n)}$}
			\DisplayProof
		\end{center}
	For K $=$ Ax observe that the atomic formulas in $\Delta'$ and $\Gamma'$ are not primed. Therefore the tautology in Ax must be between some $A_k^{(k)}$ and $B_l^{(l)}$ or some formulas in $\Gamma'$ and $\Delta'$. In the latter case we just need a single application of $S'$. In the former if $k = l$ we get the intuitionistic proof
	\begin{center}
		\AxiomC{}
		\RightLabel{Ax}
		\UnaryInfC{$A_k^{(k)},\Gamma'\Rightarrow\Delta', B_k^{(k)}$}
		\RightLabel{R$\to$}
		\UnaryInfC{$\Gamma'\Rightarrow\Delta', A_1^{(1)}\to B_1^{(1)},\dots, A_n^{(n)}\to B_n^{(n)}$}
		\DisplayProof
	\end{center}
	otherwise $k\neq l$ but $A^{(k)} = B^{(l)}$ can only be the case if $A^{(k)} = A^{(l)} = B^{(k)} = B^{(l)}$ and the above proof works thee same.
	\end{proof}

	\begin{theorem}
		A sequent $S$ in intuitionistically provable if and only if $S^\circ_\to$ is classically provable.
	\end{theorem}

	\begin{example}
		Consider Pierce's law $P := (A\to B)\to A\Rightarrow A$. This is a classic non-intuitionistic tautology. Note that it is already implicational. $P'$ is $$(A'\to B')\to A, A\to A', B\to B', \omega\to A', \omega\to B'\Rightarrow A$$ As proven this sequent is not classically valid, $A = B = B' = \omega = 0$, $A' = 1$  provides a countermodel.
		
		If we add $A\to B$ to the hypothesis set $P'$ is $$(A'\to B')\to A, A'\to B', A\to B, A\to A', B\to B', \omega\to A', \omega\to B'\Rightarrow A$$ which is classically and intuitionistically valid. Argueing in BHK terms, we can provide a terms of type $A'\to B'$ to $(A'\to B')\to A$ in order to obtain a term of type $A$, which is what is needed.
	\end{example}

	Intuitively the primed atoms may only be used in implications, i.e. as arguments to formulas of type $(A\to B)\to C$ but never as proper atoms, i.e. as arguments to $A\to B$.
	
	\bibliographystyle{plain}
	\bibliography{references}
	
\end{document}