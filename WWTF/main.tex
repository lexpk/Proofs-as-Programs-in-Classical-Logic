\documentclass[a4paper,12pt]{article}

\usepackage{url}
\usepackage{xcolor}
\usepackage[normalem]{ulem}


\begin{document}
\title{LEMONADE: Leveraging Modern Automated Deduction for Program Synthesis\\
\vspace{0.5cm}
\large{WWTF ICT 2022}}
\author{Florian Zuleger and Alexander Pluska}
\date{\today}
\maketitle

%DIESEL: Deductive Synthesis Reloaded
%DITTO: Deductive Synthesis with Automated Theorem Provers
%LEGION: Leveraging The Progress in Automated Theorem Proving for Program Synthesis
%PROGRESS: Program Synthesis via (oder despite? :)) Resolution
%LEMONADE: Leveraging Modern Automated Deduction for Program Synthesis

\begin{abstract}
  %Project Summary/Abstract (max. 1500 characters)
  %
  %Please provide an overall project summary. Clearly state the research question and the aims of the team. Describe the most innovative aspects of your proposal, and its potential relevance and impact.

  Program synthesis is a long-standing problem that crosses many areas of computer science and mathematics.
  Program synthesis holds the promise to  increase the reliability of software systems (because the synthesized program are correct by construction) and for increasing programmer productivity by enabling programmers to focus on high-level ideas instead of low-level details.

  %
  Traditionally, program synthesis has been formalized as a problem in deductive theorem proving:
  %
  Given a logical formula that states that for all inputs there exists an output such that the required specification holds, one searches for a constructive proof of this formula and then extracts a program from this proof.

  However, progress on deductive synthesis has been slow because of a lack of automated and scalable provers.
  We believe that the impressive progress in automated theorem proving finally has made automated proof search feasible.

  Thus, the main goal of our project is to leverage state-of-the-art first-order automated theorem provers (such as Vampire) for program synthesis, demonstrating the feasibility of deductive synthesis via automated theorem proving.

  Specifically, we plan to 1) adapt and extend the theory on program extraction to the specifics of modern automated theorem provers, and to 2) implement a proof-of-concept tool that demonstrates the feasibility of our approach.
\end{abstract}




\section{Introduction, background and state of the art (max. 2500 characters)}
%What is the topic and broad background of this research project? What is the state of the art in this field?
Programs are usually first written and then tested/verified.
In contrast, program synthesis allows a system developer to derive a program in one step.
At the same time, program synthesis holds the promise of freeing the programmer from low-level details and of enabling the programmer to focus on the high-level intent.
Program synthesis is a long-standing problem that crosses many areas of computer science and mathematics including programming languages, AI, formal methods and proof theory.
Traditionally, program synthesis has been formalized as a problem in deductive theorem proving~\cite{conf/ijcai/MannaW79}:
Given a logical formula that states that for all inputs there exists an output such that the required specification holds, one searches for a constructive proof of this formula and then extracts a program from this proof.
A classic route to program extraction is the \emph{Curry-Howard correspondence}~\cite{girard1989proofs} (also known as \emph{proofs-as-programs}) where one exploits the equivalence between proofs in intuitionistic logic and functional programs in order to directly extract a functional program from an intuitionistic proof.
The Curry-Howard correspondence denotes a fundamental relationship between two seemingly unrelated formalisms, namely proofs systems on one hand and computational models on the other hand.
The Curry-Howard correspondence has been a major driving force for research spanning functional programming languages, type theory, proof systems and constructive logic.
However, progress on the deductive synthesis problem has been slow for many years because of a lack of automated and scalable provers.
Instead, a recent variant of the program synthesis problem has enabled dramatic progress (we refer to the special issue~\cite{fisman2022special} and the workshop\footnote{\url{https://simons.berkeley.edu/workshops/tfcs2021-1}} for an overview):
%\footnote{https://www.cs.technion.ac.il/~shaull/SYNT2022/index.html}
The logical specification is accompanied by a \emph{syntactic template} that constrains the space of possible implementations.
The synthesis procedure then searches for/enumerates candidate implementations and either rejects these candidates by finding violating input/output pairs or accepts the generated program by verifying their correctness.
The syntactic approach is very versatile and synthesis approaches have been explored that solely start from input/output examples or that mine possible program fragments from public code repositories.
While the reported progress is impressive, we believe that the traditional deductive approach to synthesis has been unjustly neglected.


\section{Research questions, hypotheses and objectives (max. 2000 characters)}
%What are the fundamental research questions/hypotheses that the project seeks to address? What are the goals the project seeks to achieve? Provide an outline of the scientific approach that will be used to address the questions and reach the objectives. The objectives should be achievable within the duration of your project. Please provide preliminary data and any relevant research experience.

The main goal of this project is to connect the classical approach to deductive synthesis --- proofs-as-programs~\cite{girard1989proofs} ---, with modern automated theorem provers, in particular first-order solvers (including SMT solvers), which have made impressive progress over the recent years (as evidenced by yearly competitions).
%\footnote{\url{http://www.tptp.org/CASC/}})

We believe that this approach offers the following advantages:
1) First-order solvers are particularly suitable to solving \emph{quantified} formulas as needed for synthesis.
2) Proofs-as-programs represents a longstanding approach to program extraction.
That is to say, much groundwork for program extraction from proofs has been laid, we just need to adapt and extend this theory to the specifics of modern automated theorem provers.
3) We are able to take advantage of the highly sophisticated and optimized proof search strategies of first-order solvers instead of needing to define and implement \emph{ad-hoc} search strategies.
%Moreover, we will profit from any future advancements in automated theorem proving.

We believe the proposed research is very timely, because improved hardware and the progress in automated theorem proving have finally made automated proof search feasible.
We will advance the above main goal by two work packages (WPs):

(WP1) We will lay the theoretical groundwork for program extraction with automated theorem provers.
The challenge here is that the proofs-as-programs approach requires proofs in intuitionistic logic, while state-of-the-art theorem provers
utilize resolution which is an inherently classical technique.
For this, we will adapt and extend techniques from the literature that connect the Curry-Howard correspondence to classical logic.

(WP2) We will implement a proof-of-concept tool that demonstrates the feasibility of our approach.
This tool will take first-order specifications as an input, and leverage automated theorem provers to output programs that are correct by construction.
We will further experiment with techniques that relate to modern syntax-guided approaches to synthesis.


\section{Expected results, novelty and relevance (max. 2000 characters)}
%What kind of advancements are expected to be gained from the research project? Which aspects of the proposed project are especially innovative? Describe the scientific relevance and the timeliness of the research project. Please note that the development of source code, systems or protocols should not be the main goal of the project; however, they are encouraged in the context of demonstrating the practical applicability of new scientific results.

The final goal of our project is to leverage state-of-the-art first-order automated theorem provers, in particular Vampire~\cite{conf/cav/KovacsV13}, for program synthesis.
This is motivated by the recent and prospective advancements in their capabilities.
We will lay the necessary theoretical foundations as well as implement a prototype system to leverage state-of-the-art classical theorem provers for program extraction.
At the current time no such system exists.
Thus the proposed research promises to fill a longstanding gap in the state-of-the-art.

Much of the theoretical groundwork for implementing such a system is already present,
but is scattered and mostly limited to theoretical settings.
We sketch here the two main directions that connect the Curry-Howard correspondence to classical logic:
1) The Curry-Howard correspondence can be extended to classical logic using control operators~\cite{Control1}, Parigot's $\lambda\mu$-Calculus~\cite{Parigot1} or CPS-translation.
However in this process we must in general give up on some desirable properties, at the very least the realizability of $\vee$ and $\exists$.
2) Under certain conditions, classically valid proofs can be transformed into intuitionistic proofs, e.g. proofs of $\Pi_2$-sentences in Peano Arithmetic~\cite{HAPA} (this point is of particular interest for SMT).
We will adapt and specialise these techniques from the literature to the proofs produced by automated theorem provers such as Vampire:
A) We will develop the first\emph{ program extraction calculi} for \emph{resolution resp. superposition proofs}.
In particular, we need to pay careful attentiont to the normal-form transformation --- a pre-processing step --- in automated theorem provers, which also involve equivalences that are only classically valid.
B) We will identify assumptions under which our adaptions of the techniques 1) and 2) guarantee that such a program extraction is possible.
We are not aware of related results in the literature.

%This approach has been examined in different model settings and would have to be adapted and extended for our general problem.
	

\section{Methods and feasibility (max. 2000 characters)}
%Specify the methodology intended to be used in order to answer the research question(s) and objective(s).
%Describe the basic working principles and concepts, and why the chosen approach/ specific mix of approaches is the most suitable for the research question. Include a short assessment on the feasibility of the approach.
			
We now give details on (WP1):
%We will consider two approaches from the literature:

1) Parigot's $\lambda\mu$-calculus~\cite{Parigot1} gives a direct translation of classical proofs with datatypes and is essentially ready for implementation.
Unfortunately, we lose the realizability of $\exists$ and $\vee$ since non-constructive tautologies like $\neg\forall xA(x)\to \exists x\neg A(x)$ or $A\vee \neg A$ are classically provable.
In practice it is sometimes possible to eliminate the non-constructive reasoning steps from proofs~\cite{practical}.
We aim to establish stronger theoretical guarantees on when this is possible but also evaluate this approach when realizability is not ensured by a theorem.

2) Sometimes classical validity guarantees intuitionistic validity, most famously for $\Pi_2$-formulas in Arithmetic~\cite{Friedman}. There have been many variations of this result, most notably by Schwichtenberg's group~\cite{schwichtenberg}.
We will collect, adapt and extend these results with regard to our concrete setting.
This is particularly promising when applied to SMT, interpreted as a restricted subproblem of the full first-order case~\cite{reynolds2015counterexample}, or when considering altered specifications (see next point).
%We expect that in practice these approaches will be sufficient in many situations.

3) We investigate a new direction:
We aim for a translation between provers rather than proofs.
E.g., we study under what conditions classical proofs of a modified input formula allow to extract intuitionistic proofs of the actual input.
A motivating example is actually the reverse direction:
given an intuitionistic prover one can establish classical provability via double negation translation.
%A first approximation has been outlined in~\cite{RDNT}.
	
Using the results of (WP1) we will implement a synthesis system on top of Vampire (WP2), targeting a suitable functional programming language.
We will experiment with various benchmark problems %(e.g., median computation, sorting algorithms, arithmetic functions such as sqrt)
and compare our results with the SyGuS-competition.
We will use ideas from syntax-guided synthesis in order to restrict and guide proof search:
We will experiment with different formulations of arithmetic axioms, induction axioms and redundant axioms.
We will further examine if proof search is possible when only input/output pairs are given.
	
\section{Role of team members and collaborative aspects (max. 1500 characters)}
%Each team member should describe his/her role in accomplishing the goal of the team. If applicable, which different disciplines are represented in the project and which partnerships across research groups and institutions will result from the project? Which collaborative elements are essential for the project to succeed; what makes the team more than the sum of its individual contributions? A team member’s contribution should be integrated into the overall plan and should not appear merely as a resource.

\section{Potential for interdisciplinarity and application to other research fields (max. 1000 characters)}
%How will the scientific advancements gained through this project benefit other research fields that are beyond the immediate scope of the project? Describe how the knowledge, methods or technology developed in this project could further understanding or contribute to applications in other research disciplines.

%The project promises to make advancements in the design of correct of systems, automated theorem proving and proof theory.

%The complexity of modern software systems, including in particular, \emph{safety-critical systems}, continues to grow.
Traditionally quality assurance techniques rely on first building the desired system and then testing/verifying the system.
%, which may become costly or even impossible for sufficiently complex systems.
In contrast, \emph{correct-by-construction} approaches seek to develop a product that continues to be correct, from the initial design to the final product.
Advances in deductive synthesis promise to increase the applicability of the correct-by-construction approach, leading to higher reliability in software development.

Automated theorem proving has made impressive progress in recent years, demonstrated in yearly solver competitions.
We will submit our benchmarks to these competitions in order to interest solver developers in improving their tools wrt to the specific needs of deductive synthesis.
%(formulas with quantifier alternation, support for induction, etc.).

We expect that a successful prototype for program extraction with automated theorem provers will lead to a revival of proof-as-programs for automated deductive synthesis, inspiring more members of the community to investigate this approach.


\section{Potential medium-term economic / societal benefits (max. 500 characters)}
%Provide a brief statement how the results from this research project could contribute towards solving a recognised problem that affects broader society or the economy.

- As a correct-by-construction methodology, program synthesis will increase the reliability of important safety-critical systems such as avionics, cars, medical devices, etc.

\noindent - Enabling programmers to focus on high-level ideas instead of low-level details will increase programmer productivity.

\noindent - Non-expert/ end-user programmers are also expected to promise from automated synthesis by only needing to communicate their intent to the synthesis engine (instead of writing a whole program).


\section{Potential ethical aspects (max. 500 characters)}
%If required, information should be given with respect to the ethical approval. Is an ethical approval available that covers the prospective use of data and the planned analysis? If not, state whether an additional approval is required.

Due to the focus on basic research and the mathematical and formal methodology, the proposed research does not entail ethical aspects beyond the usual mandate of good scientific practice.
Similarly, the proposed research does not entail sex-specific or gender-related aspects.

\section{Further Needed Stuff}

Key references (max. 10 citations, max. 1250 characters)
Cite a maximum of 10 of the most relevant background publications for the proposal. There is no required citation format, nevertheless publications should be cited in a way that allows the reader to easily retrieve the key information.

Schedule/ Project overview (1 page Gantt chart, PDF upload)
Upload a Gantt chart to give an overview of the milestones to be achieved during the project period. Specify the time periods and the respective PIs responsible for each milestone.

\bibliographystyle{plain} % We choose the "plain" reference style
\bibliography{literature} % Entries are in the refs.bib file

%[1] Ulrich Berger, Wilfried Buchholz, and Helmut Schwichtenberg. Refined
%program extraction from classical proofs. 2002.
%
%[2] Dana Fisman, Rishabh Singh, and Armando Solar-Lezama. Special issue
%on syntax-guided synthesis preface. 2022.
%
%[3] Harvey Friedman. Classically and intuitionistically provably recursive
%functions. 1978.
%
%[4] Jean-Yves Girard, Paul Taylor, and Yves Lafont. Proofs and types. 1989.
%
%[5] Timothy G Griffin. A formulae-as-type notion of control. 1989.
%
%[6] Ulrich Kohlenbach. The Friedman A-translation. 2008.
%
%[7] Yevgeniy Makarov. Practical program extraction from classical proofs.  2006.
%
%[8] Zohar Manna and Richard Waldinger. A deductive approach to program
%synthesis. 1979.
%
%[9] Michel Parigot. λμ-calculus: an algorithmic interpretation of classical
%natural deduction. 1992.
%
%[10] Andrew Reynolds, Morgan Deters, Viktor Kuncak, Cesare Tinelli, and
%Clark Barrett. Counterexample-guided quantifier instantiation for synthesis
%in SMT. 2015.

\end{document} 